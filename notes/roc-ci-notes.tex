\documentclass{article}

\usepackage{amsmath}

\title{Notes about creating confidence bands for receiver operating characteristic curves}

\author{Yuriy Sverchkov}

\begin{document}
\maketitle

We follow Tilbury et al.(2000) in estimating confidence bands for receiver operating characteristic (ROC) curves.
Their methods computes an probability surface for each ROC curve point.
The idea is, for each point $\hat x, \hat y$ on the ROC curve, for each point $x, y$ on the ROC axes, to compute the exact probability of getting the $\hat x$ false positive rate (FPR) and $\hat y$ true positive rate (TPR) given true values $x, y$.

Particularly, they show that the FPR and TPR can be treated as independent, given that, for the FPR, if we observed $a_1$ true negatives and $a_0$ false positives, the probability of the observation given that the true FPR is $x: q < x < r$, is proportional to:
\begin{equation}
 \int_q^r x^{a_0} (1-x)^{a_1} dx = \int_0^r x^{a_0} (1-x)^{a_1} dx - \int_0^r x^{a_0} (1-x)^{a_1} dx
\end{equation}
and
\begin{equation}
\int_0^s x^{a_0} (1-x)^{a_1} dx = \Gamma(a_0+1) \Gamma(a_1+1) \sum_{k=0}^{a_1} \frac{(1-s)^k s^{a_0 + a_1 + 1 - k}}{\Gamma(k+1) \Gamma(a_0 + a_1 + 2 - k)}
\end{equation}
where the normalization constant for this distribution is clearly
\begin{equation}
\int_0^1 x^{a_0} (1-x)^{a_1} dx = B(a_0+1, a_1+1) = \frac{\Gamma(a_0+1) \Gamma(a_1+1)}{\Gamma(a_0+a_1+2)} \text{.}
\end{equation}
Therefore,
\begin{equation}
P(a_0, a_1 | x<s) = \Gamma(a_0+a_1+2) \sum_{k=0}^{a_1} \frac{(1-s)^k s^{a_0 + a_1 + 1 - k}}{\Gamma(k+1) \Gamma(a_0 + a_1 + 2 - k)}
\end{equation}
\end{document}